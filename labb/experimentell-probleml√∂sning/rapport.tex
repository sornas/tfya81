\documentclass[a4paper,12pt]{article}

\usepackage[T1]{fontenc}
\usepackage[utf8]{inputenc}

\usepackage{amssymb, amsmath, amsthm}
\usepackage{systeme}

\usepackage{subfig}
\usepackage{pgfplots}
\usepackage{pgfplotstable}

\usepackage[swedish]{babel}
\usepackage{lipsum}
\usepackage{parskip}

\newcommand*{\dd}{\mathrm{d}}
%\newcommand*{\ln}{\mathrm{ln}}

\begin{document}

\section*{Sammanfattning}
\lipsum[1]

\clearpage

\tableofcontents
\clearpage

\section{Inledning}
\lipsum[1]

\section{Metod}

%\subsection{Experimentets utförande}
En metalltråd fästes med hjälp av en tving-liknande klämma i ena änden och läts
i andra änden hänga fritt. I den fria änden fästes mitten av en stång med en
insex-nyckel. På stångens båda sidor skruvades vikter på med lika avstånd från
mittpunkten av stången. Vid vardera mätning roterades stången en bit och läts
sedan pendla fritt. Pendeltiden mättes för tio pendelutslag åt gången för att
minska osäkerheten.

\subsection{Ingående variabler}

\begin{table}[h!]
  \caption{Storheter}
  \label{tab:storheter}
  \begin{tabular} {| l | l | l | l |}
    \hline
    \textbf{Beskrivning} & \textbf{Variabel} & \textbf{Enhet} & \textbf{Dimension} \\\hline
    Trådens längd & $l$ & [m] & L \\\hline
    Trådens diameter & $d$ & [m] & L \\\hline
    Viktens massa & $m$ & [kg] & M \\\hline
    Viktens avstånd från mitten & $a$ & [m] & L \\\hline
    Trådens skjuvmodul & $G$ & [Pa] $=$ [kg $\mathrm{m}^{-1}$ $\mathrm{s}^{-2}$] & M $\mathrm{L}^{-1}$ $\mathrm{T}^{-2}$ \\\hline
    Pendeltiden & $T$ & [s] & T \\\hline
    Konstant med dimension & $c_1$ & [?] & ? \\\hline
    Konstant utan dimension & $c_2$ && \\\hline
  \end{tabular}
\end{table}

Trådens diameter mättes med en mikrometer. Övriga längder mättes med måttband.
Massan mättes med våg, skjuvmodulerna gavs i laborationsinstruktionen och
tiden mättes med stoppur.

\subsection{Hypotes}

...

I övrigt antogs sambandet vara multiplikativt.

\begin{equation}
  T = L^a \cdot d^b \cdot (m^c \cdot r^d + c_1)^e \cdot G^f \cdot c_2
\end{equation}

\subsection{Dimensionsanalys}

Givet ovanstående ansats ficks följande ekvationssystem gällande exponenterna.

\begin{equation}
  %\left\{
  \begin{array}{rrcc}
    \text{m:} & \hspace{1em} a + b + de - f & = & 0 \\
    \text{kg:} & c e + f & = & 0 \\
    \text{s:} & 2f & = & 1
  \end{array}
  \label{eq:system}
\end{equation}

$f = \frac{-1}{2}$ antogs som uppenbar. $a$ och $b$ ficks sedan genom
linjäriseringar och $c$, $d$ och $e$ ficks genom en ansats.

\section{Resultat}

\subsection{Linjärisering}

19 mätningar (mätserie~1) gjordes där $L$ var den enda variabeln som varierades
varpå följande linjärisering genomfördes.

\begin{align}
  \ln T &= \ln (L^a \cdot d^b \cdot (m^c \cdot r^d + c_1)^e \cdot G^f \cdot c_2) \nonumber \\
  \ln T &= \ln (L^a) + \ln (d^b \cdot (m^c \cdot r^d + c_1)^e \cdot G^f \cdot c_2) \nonumber \\
  \ln T &= a\ln L + ln (d^b \cdot (m^c \cdot r^d + c_1)^e \cdot G^f \cdot c_2)  \label{eq:lin_La}
\end{align}

Dessutom gjordes 15 mätningar (mätserie~2) där $d$ var den enda variabeln som varierades och
följande linjärisering gjordes.

\begin{align}
  \ln T &= \ln (L^a \cdot d^b \cdot (m^c \cdot r^d + c_1)^e \cdot G^f \cdot c_2) \nonumber \\
  \ln T &= \ln (d^b) + \ln (L^a \cdot (m^c \cdot r^d + c_1)^e \cdot G^f \cdot c_2) \nonumber \\
  \ln T &= b\ln d + \ln (L^a \cdot (m^c \cdot r^d + c_1)^e \cdot G^f \cdot c_2)  \label{eq:lin_db}
\end{align}

Ekvationerna \eqref{eq:lin_La} och \eqref{eq:lin_db} ritades upp enligt
figur~\ref{fig:lin_La} och figur~\ref{fig:lin_db}.

\begin{figure}
  \begin{tikzpicture}
    \begin{axis} [
      ylabel=$\ln T$,
      xlabel=$\ln L$,
      xmin=-2.1, xmax=-0.9,
      ymin=-0.45, ymax=0.14,
      legend pos = outer north east,
      ]
      \addplot+ [mark=none] {0.4601*x + 0.5066};
      \addlegendentry {$0.46x + 0.51$};
      \addplot+ [only marks,mark=*,color=blue,mark options={fill=blue}] table [col sep=comma, x index=2, y index=3] {data/var_l.csv};
    \end{axis}
  \end{tikzpicture}
  \caption{Linjärisering med avseende på $L$}
  \label{fig:lin_La}
\end{figure}

\begin{figure}
  \begin{tikzpicture}
    \begin{axis} [
      ylabel=$\ln T$,
      xlabel=$\ln d$,
      xmin=-6.45, xmax=-5.95,
      ymin=1.85, ymax=2.65,
      legend pos = outer north east,
      ]
      \addplot+ [mark=none,domain=-7:-5] {-1.897*x - 9.4668};
      \addlegendentry {$-1.9x - 9.5$};
      \addplot+ [only marks,mark=*,color=blue,mark options={fill=blue}] table [col sep=comma, x index=2, y index=3] {data/var_d.csv};
    \end{axis}
  \end{tikzpicture}
  \caption{Linjärisering med avseende på $d$}
  \label{fig:lin_db}
\end{figure}

Eftersom lutningen i figur~\ref{fig:lin_La} beskrev exponenten $a$ och lutningen
i figur~\ref{fig:lin_db} beskrev exponenten $b$ antogs $a = 0.5$ och $b = -2$.

\subsection{Ansättning}

På grund av additionen under exponenten $e$ kunde varken $c$, $d$ eller $e$
beräknas genom linjärisering. Istället gjordes en omskrivning med hjälp av
ekvationssystemet \eqref{eq:system} och ett värde på $c$ ansattes.

\begin{align}
  T &= L^a \cdot d^b \cdot (m^c \cdot r^d + c_1)^e \cdot G^f  \cdot c_2\nonumber \\
  T &= D (m^c \cdot r^d + c_1)^e \nonumber \\
  T &= D (m^c \cdot r^{2c} + c_1)^{1/2c} \label{eq:use_system} \\
  T^{2c} &= D^{2c} (m^c \cdot r^{2c} + c_1) \nonumber \\
  T^{2c} &= (D^2m)^c r^{2c} + D^{2c}c_1 \label{eq:lin_c}
\end{align}

I ekvation~\eqref{eq:use_system} användes ekvationsystemet så alla exponenter
beskrevs utifrån $c$.

I ekvation~\eqref{eq:lin_c} ansattes sedan $c = 1$. 35 mätningar (mätserie~3)
utfördes där $r$ var den enda variabeln som varierade och figur~\ref{fig:lin_c}
ritades ut för att kontrollera hypotesen.

\begin{figure}[h!]
  \begin{tikzpicture}
    \begin{axis} [
      xticklabel style={
        /pgf/number format/fixed,
        /pgf/number format/precision=3
      },
      scaled x ticks=false,
      ylabel=$T^2$,
      xlabel=$r^2$,
      xmin=-0.005, xmax=0.063,
      ymin=-0.1, ymax=1.8,
      legend pos = outer north east,
      ]
      \addplot+ [mark=none, domain=-0.01:0.1] {22.9564*x + 0.3024};
      \addlegendentry{$23.0x + 0.3$};
      \addplot+ [only marks,mark=*,color=blue,mark options={fill=blue}] table [col sep=comma, x index=2, y index=3] {data/var_r.csv};
    \end{axis}
  \end{tikzpicture}
  \caption{Ansättningen $c = 1$}
  \label{fig:lin_c}
\end{figure}

På grund av den höga linjäriteten (framförallt för $r$ nära 0) antogs $c = 1$
vara en korrekt ansättning. Ekvationssystemet löstes sedan fullständigt.

\begin{align}
  ce - \frac{1}{2} = 0 \quad &\Leftrightarrow \quad ce = \frac{1}{2} \quad \Leftrightarrow \quad e = \frac{1}{2c} = \frac{1}{2} & \\[0.5em]
  \frac{1}{2} - 2 + de - \frac{1}{2} = 0 \quad &\Leftrightarrow \quad de = 2 \quad \Leftrightarrow \quad d = 2c = 2 &
\end{align}

\subsection{Bestämning av konstanter}

Konstanterna $c_1$ och $c_2$ bestämdes genom följande linjärisering.
Mätserie~3 återanvändes och figur~\ref{fig:lin_c1_c2} ritades ut.

\begin{align}
  T &= L^{1/2} \cdot d^{-2} \cdot (m r^2 + c_1)^{1/2} \cdot G^{-1/2} \cdot c_2 \nonumber \\
  T^2 &= L \cdot d^{-4} \cdot (m r^2 + c_1) \cdot G^{-1} \cdot c_2^2 \nonumber \\
  \underbrace{\frac{T^2 d^4 G}{L}}_y &= \underbrace{c_2^2 \vphantom{\frac{T^2 d^4 G}{L}}}_k \underbrace{m r^2 \vphantom{\frac{T^2 d^4 G}{L}}}_x + \underbrace{c_1c_2^2 \vphantom{\frac{T^2 d^4 G}{L}}}_m
\end{align}

\begin{figure}[h!]
  \begin{tikzpicture}
    \begin{axis} [
      xticklabel style={
        /pgf/number format/fixed,
        /pgf/number format/precision=3
      },
      scaled x ticks=false,
      ylabel=$y$,
      xlabel=$x$,
      xmin=-0.0005, xmax=0.0045,
      ymin=-0.5, ymax=4.9,
      legend pos = outer north east,
      ]
      \addplot+ [mark=none,domain=-0.0005:0.005] {880.998*x + 0.787};
      \addlegendentry{$881 (\pm 4) \cdot x + 0.787 (\pm 0.007)$};
      \addplot+ [only marks,mark=*,color=blue,mark options={fill=blue}] table [col sep=comma, x index=0, y index=1] {data/c1_c2.csv};
    \end{axis}
  \end{tikzpicture}
  \caption{Linjärisering för att bestämma $c_1$ och $c_2$}
  \label{fig:lin_c1_c2}
\end{figure}

\begin{align}
  k = c_2^2 \quad &\Leftarrow \quad c_2 = \sqrt{k} \approx 29.7\\
  m = c_1c_2^2 \quad &\Leftarrow \quad c_1 = \frac{m}{k} \approx 0.00089
\end{align}

\subsection{Felanalys}

Följande fel uppskattades i avläsningen av de olika storheterna.

\begin{tabular}{|l|l|}
  \hline
  \textbf{Storhet} & \textbf{Mätfel} \\\hline
  Trådens längd ($L$) & $\pm 0.5 \text{mm}$ \\\hline
  Trådens diameter ($d$) & $\pm ?? \text{mm}$ \\\hline
  Viktens massa ($m$) & $\pm ?? \text{g}$ \\\hline
  Viktens avstånd från mitten ($a$) & $\pm 0.5 \text{mm}$ \\\hline
  Pendeltiden ($T$) & $\pm 0.05 \text{s}$ \\\hline
\end{tabular}

Osäkerheten i $c_1$ och $c_2$ bedömdes med felfortplantningsformeln.

Följande ekvationer ställdes upp.

\begin{align}
  c_2(k) &= \sqrt{k} \\
  c_1(k, m) &= \frac{m}{k}
\end{align} 

Utifrån dessa ställdes följande feluppskattningar upp.

\begin{align}
  s(c_2) &= \frac{\dd c_2}{\dd k} \, s(k) = \frac{1}{2\sqrt{k}} \, s(k) \approx ...\\
  s(c_1) &= \sqrt{\left( \frac{\partial c_1}{\partial k} \, s(k) \right)^2 + \left( \frac{\partial c_1}{\partial m} \, s(m) \right)^2} \nonumber \\
         &= \sqrt{\left( \frac{-m}{k^2} \, s(k) \right)^2 + \left( \frac{1}{k} \, s(m) \right)^2} \approx ...
\end{align}

\section{Slutsats och diskussion}
\lipsum[1-2]

\clearpage
\appendix
\section{Bilagor}

\end{document}
